\documentclass[11pt,a4paper]{article}
\usepackage[utf8]{inputenc}
\usepackage[T1]{fontenc}
\usepackage[french]{babel}
\usepackage{graphicx}
\usepackage{geometry}
\geometry{margin=2.5cm}
\usepackage{amsmath}
\usepackage{listings}
\usepackage{xcolor}

\title{Projet : Formes Géométriques \\ Groupe 22}
\author{BAMPIRE}
\date{\today}

\definecolor{codegray}{gray}{0.9}
\lstset{
  backgroundcolor=\color{codegray},
  basicstyle=\ttfamily,
  frame=single,
  language=Python
}

\begin{document}
\maketitle

\section{Introduction}
Ce rapport présente deux approches de modélisation orientée objet appliquées aux formes géométriques : une version utilisant l’héritage et une version basée sur la composition.

\section{Version avec héritage}
\subsection{Structure}
Toutes les classes héritent d’une classe abstraite \texttt{Forme}. Voici un extrait de code :

\begin{lstlisting}
class Forme:
    def surface(self):
        raise NotImplementedError()
    def perimetre(self):
        raise NotImplementedError()
\end{lstlisting}

\subsection{Diagramme UML}
\includegraphics[width=\textwidth]{uml_heritage.png}

\subsection{Tests}
...

\section{Version avec composition}
\subsection{Structure}
Les classes sont toutes indépendantes. Exemple :

\begin{lstlisting}
class Rectangle:
    def __init__(self, longueur, largeur):
        self.longueur = longueur
        self.largeur = largeur
\end{lstlisting}

\subsection{Diagramme UML}
\includegraphics[width=\textwidth]{uml_composition.png}

\subsection{Tests}
...

\section{Analyse comparative}
...

\section{Conclusion}
...

\end{document}
